\begin{center}{\large{
      Tuesday, June 5, 9:45 AM - 10:30 AM \\
      \textbf{IP1: Flexible methodology for image segmentation}\\
Raymond H. Chan, Department of Mathematics, The Chinese University of Hong Kong,  Hong Kong}}\\
\end{center}

\begin{center}{\large{
Tuesday, June 5, 10:30 AM - 11:15 AM \\
\textbf{IP2: The expanding role of inverse problems in informing climate science and policy}\\
Anna Michalak, Department of Global Ecology, Carnegie Institution for Science, Stanford, USA}}\\
\end{center}

\begin{center}{\large{
Wednesday, June 6, 8:15 AM - 9:00 AM \\
\textbf{IP3: Fake ID documents and counterfeited products: overview of image analysis techniques to fight them}\\
Clarissa Mandridake, Surys Group, France}}\\
\end{center}

\begin{center}{\large{
Thursday, June 7, 8:15 AM - 9:00AM \\
\textbf{IP4: Fast analog to digital compression for high resolution imaging}\\
Yonina Eldar, Department of EE Technion, Israel Institute of Technology, Haifa, Israel
}}\\
\end{center}

\begin{center}{\large{
Thursday, June 7, 1:00 PM - 1:45 PM \\
\textbf{IP5: Image Segmentation and Understanding: A Challenge for Mathematicians}\\
Christoph Schn\"{o}rr, Institute of Applied Mathematics, University of Heidelberg, Germany}}\\
\end{center}

\begin{center}{\large{
Friday, June 8, 8:15 AM - 9:00 AM \\
\textbf{IP6: Linearly-convergent stochastic gradient algorithms}\\
Francis Bach, Departement d'Informatique de l'Ecole Normale Superieure Centre de Recherche INRIA de Paris, France}}
\end{center}


