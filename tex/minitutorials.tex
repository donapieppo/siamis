%--------------MINITUTORIALS
\newpage
\section*{Minitutorials}
\addcontentsline{toc}{section}{Minitutorials}


%---------MT1
\begin{center}\textbf{Wednesday, June 6, 10:00 AM - 12:00 PM \\}\end{center}
\vspace{2cm}
\begin{center}
\textbf{MT1: Computational Uncertainty Quantification for Inverse Problems}\\
Organizer: John Bardsley, Department of Mathematical Sciences, The University of Montana, MT, USA\\
Building A, Room A
\end{center}
\vspace{2cm}

\begin{wrapfloat}{figure}{o}{0pt}
\includegraphics[scale=1.2]{John_Bardsley.jpg}
\end{wrapfloat}
Dr. Johnathan M. Bardsley is Professor of Mathematics at the University of Montana (UM) in Missoula, Montana, USA. He received his PhD from Montana State University in 2002 under the direction of Professor Curtis R. Vogel, with a dissertation focused on computational inverse problems. He then spent one year as a post-doc at the Statistical and Applied Mathematical Sciences Institute, under the direction of Professor H. Thomas Banks. He began his current job at UM in 2003, and since then has spent two years abroad as a visiting Professor: first at the University of Helsinki in Finland in 2006-07; and then at the University of Otago in New Zealand in 2010-11. Dr. Bardsley has published over 45 refereed journal articles and has given many presentations on his research around the world. He also organized Montana Uncertainty Quanitifcation, a conference/workshop that took place at UM in June 2015. Dr. Bardsley?s current research is focused, broadly, on uncertainty quantification for inverse problems, and more specifically, on the development of Markov chain Monte Carlo methods for sampling from posterior distributions that arise in both linear and nonlinear inverse problems. He has a forthcoming book, titled Computational Methods and Uncertainty Quantification for Inverse Problems, that will be published by SIAM
\newpage

%----MT2
\vspace{2cm}
\begin{center}
\textbf{MT2: Automated 3D reconstruction from satellite images}\\
Organizers: Gabriele Facciolo, Carlo de Franchis and Enric Meinhardt-Llopis, Centre de math\'ematiques et de leurs applications [CMLA] - Ecole Normale Sup\'erieure de Cachan, France\\
Building A, Room B
\end{center}

\vspace{2cm}
\begin{wrapfloat}{figure}{o}{0pt}
\includegraphics[scale=0.65]{gabriele_facciolo.jpg}
\end{wrapfloat}
Gabriele Facciolo received his B.Sc. and M.Sc. in computer science from Universidad de la Republica del Uruguay, and his Ph.D. (2011) from Universitat Pompeu Fabra under the supervision of Vicent Caselles. During his thesis he contributed to a pioneering mathematical formalization of the image inpainting problem, and a formulation of temporally consistent video editing robust to illumination changes. He joined Jean-Michel Morel's group at the \'{E}cole Normale Sup\'erieure Paris-Saclay in 2011 where he is currently associate research professor. He has participated in many industrial projects, creating image processing algorithms and transferring technology with the CNES, Schlumberger, DxO Labs, and the foundation BarcelonaMedia. He has more than ten years of experience designing algorithms for remote sensing applications and collaborating with the French Space Agency (CNES) as part of the MISS project (Mat\'ematiques de l'Imagerie St\'er\'eoscopique Spatiale). The 3D reconstruction algorithms and the satellite stereo pipeline (github.com/MISS3D/s2p) he and his team have created within the CMLA have been adopted as the CNES?s official stereo pipeline. He and his team also won the 2016 IARPA Multi-View Stereo 3D Mapping Challenge. He is one of the founding Editors of IPOL (www.ipol.im), the first journal publishing articles associated to online executable algorithms.
%-----MT3
\newpage
\vspace{2cm}
\begin{center}
\textbf{MT3: Regularization of Inverse Problem}\\
Organizer: Otmar Scherzer, Computational Science Center, University of Vienna\\
Building A, Room C and Room D
\end{center}


\vspace{2cm}
\begin{wrapfloat}{figure}{o}{0pt}
\includegraphics[scale=0.65]{otmar_scherzer.jpg}
\end{wrapfloat}
Otmar Scherzer received his PhD and Habilitation from the University of Linz (Austria) in 1990, 1995, respectively. He was a postdoc researcher at Texas A\&M University and the University of Delaware. He held professorships at the Ludwig Maximilian University Munich, University of Bayreuth, University of Innsbruck before he became professor at the University of Vienna, where he is now the head of the Computational Science Center. In addition he is research group leader of the ``Imaging and Inverse Problems Group?? of the Radon Institute of Computational and Applied Mathematics (RICAM) in Linz, which is an institute of the Austrian Academy of Sciences. Otmar Scherzer is an expert in regularization theory and mathematical imaging. He has about 200 publications in leading journals in these fields and is editor of about 10 journals and book series, including SIAM J. imaging Sciences. Moreover, he published two monographs, and edited several books, including the Handbook of Mathematical Imaging in three volumes. In 1991 he received the Theodor K\"orner Prize, the Prize of the Austrian Mathematical Society, the science prize of Tyrol, and in 1999 the START-prize of the Austrian Science Foundation, which is the highest award for young Austrian scientists in Austria. From 2010 to 2017 he has been Vice-president of the Inverse Problems International Association (IPIA).


