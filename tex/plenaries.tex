%----------------INVITED PLENARY PRESENTATIONS

\chapter*{Invited Plenary Speakers and Biography}
\addcontentsline{toc}{section}{Invited Plenary Speakers and Biography}
{\small{Invited Plenary Presentations  IP1 and IP2 on Tuesday, June 5, will take place in Aula Santa Lucia, while the other Invited Presentation will take place in Building A, Rooms A and B in via Belmeloro 14}}
%\vspace{2cm}

%---IP1
\newpage\vspace{2cm}
\begin{center}{\Large{
Tuesday, June 5, 9:45 AM - 10:30 AM \\
\textbf{IP1: Linearly-convergent stochastic gradient algorithms}\\
Francis Bach, Departement d'Informatique de l'Ecole Normale Superieure Centre de Recherche INRIA de Paris, PARIS, France}}
\end{center}
\vspace{1cm}

\begin{wrapfloat}{figure}{o}{0pt}
  \includegraphics[scale=1.25]{{../app/assets/images/speakers/bardsleyj@mso.umt.edu}.jpg}
\end{wrapfloat}

Francis Bach is a researcher at Inria, leading since 2011 the machine learning project-team, which is part of the Computer Science Department at Ecole Normale Sup\'erieure. He graduated from Ecole Polytechnique in 1997 and completed his Ph.D. in Computer Science at U.C. Berkeley in 2005, working with Professor Michael Jordan. He spent two years in the Mathematical Morphology group at Ecole des Mines de Paris, then he joined the computer vision project-team at Inria/Ecole Normale Sup\'erieure from 2007 to 2010. Francis Bach is primarily interested in machine learning, and especially in graphical models, sparse methods, kernel-based learning, large-scale convex optimization, computer vision and signal processing. He obtained in 2009 a Starting Grant and in 2016 a Consolidator Grant from the European Research Council, and received in 2012 the Inria young researcher prize. In 2015, he was program co-chair of the International Conference in Machine learning (ICML).\\\\

\textbf{Abstract:}\\

Many machine learning and signal processing problems are traditionally cast as convex optimization problems where the objective function is a sum of many simple terms. In this situation, batch algorithms compute gradients of the objective function by summing all individual gradients at every iteration and exhibit a linear convergence rate for strongly-convex problems. Stochastic methods rather select a single function at random at every iteration, classically leading to cheaper iterations but with a convergence rate which decays only as the inverse of the number of iterations. In this talk, I will present the stochastic averaged gradient (SAG) algorithm which is dedicated to minimizing finite sums of smooth functions; it has a linear convergence rate for strongly-convex problems, but with an iteration cost similar to stochastic gradient descent, thus leading to faster convergence for machine learning and signal processing problems. I will also mention several extensions, in particular to saddle-point problems, showing that this new class of incremental algorithms applies more generally.

%---IP2
\newpage\vspace{2cm}
\begin{center}{\Large{
Tuesday, June 5, 10:45 AM - 11:30 AM \\
\textbf{IP2: Flexible methodology for image segmentation}\\
Raymond H. Chan, Department of Mathematics, The Chinese University of Hong Kong,  Hong Kong}}
\end{center}
\vspace{1cm}

\begin{wrapfloat}{figure}{o}{0pt}
\includegraphics[scale=0.75]{Raymond_Chan.jpg}
\end{wrapfloat}

Raymond Chan obtained his PhD degree from The Courant Institute of Mathematical Sciences in 1985. He is now the Chairman of the Mathematics Department at The Chinese University of Hong Kong. He won a Leslie Fox Prize in 1989; a Feng Kang Prize in 1997; a Morningside Award in 1998; and 2011 Higher Education Outstanding Scientific Research Output Awards from the Ministry of Education in China. He was elected a SIAM Fellow in 2013. He has published 120 journal papers and has been in the ISI Science Citation List of Top 250 Highly-Cited Mathematicians in the world since 2004. Chan has served on the editorial boards of many journals, including: Journal of Mathematical Imaging and Vision, Journal of Scientific Computing, Numerical Linear Algebra with Applications, SIAM Journal on Imaging Sciences, and SIAM Journal on Scientific Computing. 

%---IP3
\newpage\vspace{2cm}
\begin{center}{\Large{
Wednesday, June 6, 8:30 AM - 9:15 AM \\
\textbf{IP3: Fast analog to digital compression for high resolution imaging}\\
Yonina Eldar, Department of EE Technion, Israel Institute of Technology, Haifa, Israel
}}\end{center}
\vspace{1cm}


\begin{wrapfloat}{figure}{o}{0pt}
\includegraphics[scale=0.18]{Yonina_Eldar.jpg}
\end{wrapfloat}
Yonina C. Eldar is a Professor in the Department of Electrical Engineering at the Technion, Israel Institute of Technology, Haifa, where she holds the Edwards Chair in Engineering. 
She is also a Research Affiliate with the Research Laboratory of Electronics at MIT and a Visiting Professor at Duke University, and was a Visiting Professor at Stanford University. 
She received the B.Sc. degree in physics and the B.Sc. degree in electrical engineering both from Tel-Aviv University (TAU), Tel-Aviv, Israel, in 1995 and 1996, respectively, and the Ph.D. degree in electrical engineering and computer science from the Massachusetts Institute of Technology (MIT), Cambridge, in 2002.
She has received many awards for excellence in research and teaching, as
the IEEE Signal Processing Society Technical Achievement Award (2013), the IEEE/AESS Fred Nathanson Memorial Radar Award (2014) and the IEEE Kiyo Tomiyasu Award (2016).
She was a Horev Fellow of the Leaders in Science and Technology program at the Technion and an Alon Fellow. 
She received the Michael Bruno Memorial Award from the Rothschild Foundation, the Weizmann Prize for Exact Sciences, the Wolf Foundation Krill Prize for Excellence in Scientific Research, the Henry Taub Prize for Excellence in Research (twice), the Hershel Rich Innovation Award (three times), the Award for Women with Distinguished Contributions, the Andre and Bella Meyer Lectureship, the Career Development Chair at the Technion, the Muriel \& David Jacknow Award for Excellence in Teaching, and the Technion's Award for Excellence in Teaching (two times). 
She received several best paper awards and best demo awards together with her research students and colleagues and was selected as one of the 50 most influential women in Israel.
She is the Editor in Chief of Foundations and Trends in Signal Processing, a member of several IEEE Technical Committees and Award Committees, an IEEE Fellow, and a EURASIP Fellow.
She is also a member of the Young Israel Academy of Science and was a member of the Israel Committee for Higher Education

%--IP4
\newpage\vspace{2cm}
\begin{center}{\Large{
Wednesday, June 6, 1:30 PM - 2:15 PM \\
\textbf{IP4: Fake ID documents and counterfeited products: overview of image analysis techniques to fight them}\\
Clarissa Mandridake, Surys Group, France}}
\end{center}
\vspace{1cm}

\begin{wrapfloat}{figure}{o}{0pt}
\includegraphics[scale=0.12]{Clarissa_Mandridake.jpg}
\end{wrapfloat}
Mrs Clarisse Manjary Mandridake received his PhD in Image and Signal Processing from the University of Bordeaux I, France, for her works on bi-dimensional signal decomposition applied to classification of textured images, done in Laboratoire Automatique Productique et Traitement du Signal, and in closed connection with ARIANA Project in INRIA Sophia-Antipolis. She joined the research team of Advestigo for her postdoc year in 2002. As a researcher at Advestigo and later at Hologram Industries (now renamed SURYS), she developed technologies for the representation, indexation and search of images and videos in large scale databases. She is now in charge of the coordination of the research project for the SURYS digital labs and animates the scientist partnerships with University labs. Her expertise covers Applied Mathematics, image characterization, fingerprinting and authentication on various physical supports, from ID documents to smartlabels. More recently, his area of interest is to contribute to technological innovation for use by poor countries or developing countries in order to help them put in place what is called "good governance". It is a sine qua non Condition for any future economic development.


%--IP5
\newpage\vspace{2cm}
\begin{center}{\Large{
Thursday, June 7, 1:30 PM - 2:15 PM \\
\textbf{IP5: The expanding role of inverse problems in informing climate science and policy}\\
Anna Michalak, Department of Global Ecology, Carnegie Institution for Science, Stanford, CA, USA}}
\end{center}
\vspace{1cm}

\begin{wrapfloat}{figure}{o}{0pt}
\includegraphics[scale=0.25]{Anna_Michalak.jpg}
\end{wrapfloat}
Dr. Anna M. Michalak is a faculty member in the Department of Global Ecology of the Carnegie Institution for Science in Stanford, California, and an Associate Professor in the Department of Earth System Science at Stanford University. Prior to joining Carnegie, she was the Frank and Brooke Transue Faculty Scholar and Associate Professor at the University of Michigan, Ann Arbor. Her research interests primarily lie in two areas. She explores the impacts of climate change and extreme events on freshwater and coastal water quality via influences on nutrient delivery to, and on conditions within, water bodies. She also studies the cycling and emissions of greenhouse gases at the Earth surface at regional to global scales ? scales directly relevant to informing climate science and policy ? primarily through the use of atmospheric observations that provide the clearest constraints at these critical scales. She is the recipient of numerous awards, including the Presidential Early Career Award for Scientists and Engineers (nominated by NASA), the NSF CAREER award, the Association of Environmental Engineering and Science Professors Outstanding Educator Award, and the Leopold Fellowship in environmental leadership. Dr. Michalak holds a B.Sc. from the University of Guelph, Canada, and M.S. and Ph.D. degrees from Stanford.

%--IP6
\newpage\vspace{2cm}
\begin{center}{\Large{
Friday, June 8, 1:30 PM - 2:15 PM \\
\textbf{IP6: Image Segmentation and Understanding: A Challenge for Mathematicians}\\
Christoph Schn\"{o}rr, Institute of Applied Mathematics, University of Heidelberg, Germany}}
\end{center}
\vspace{1cm}

\begin{wrapfloat}{figure}{o}{0pt}
\includegraphics[scale=0.25]{Schnoerr.jpg}
\end{wrapfloat}
Christoph Schn\"{o}rr received his degrees from the Technical University of Karlsruhe (today: Karlsruhe Institute of Technology) and the University of Hamburg, respectively. He worked as a researcher at the Fraunhofer Institut of Information and Data Processing in Karlsruhe before moving to the University of Hamburg. In 1998, he became full professor at the University of Mannheim, where he set up and directed the Computer Vision and Pattern Recognition Group. He moved to the Heidelberg University in 2008 where he is heading the Image and Pattern Analysis Group at the Institute of Applied Mathematics, which also is member of the Interdisciplinary Center for Scientific Computing.

Christoph Schn\"{o}rr has been coordinating 2010-2018 a research training group focusing on probabilistic graphical models and its applications to image analysis, funded by the German Science Foundation. He is one of 4 directors of the Heidelberg Collaboratory for Image Processing that implements and explores novel ways of combining basic strategic research in academia and research labs in industry, as part of the excellence initiative of the Heidelberg University. He served 2005-2014 as co-editor in chief of the International Journal of Computer Vision and currently as associate editor for the Journal of Mathematical Imaging and Vision and the SIAM Journal of Imaging Science.

His research interests include mathematical models of image analysis and numerical optimisation.



