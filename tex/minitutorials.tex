%--------------MINITUTORIALS
\begin{center}\textbf{Wednesday, June 6, \\9:30 AM - 11:30 AM \\Room B}\end{center}
\begin{center}
\textbf{MT1: Computational Uncertainty Quantification for Inverse Problems}\\
Organizer: John Bardsley, Department of Mathematical Sciences, The University of Montana, MT, USA
\end{center}

\vspace{1cm}
\begin{center}\textbf{Thursday, June 7, \\9:30 AM - 11:30 AM \\Room B}\end{center}
%\vspace{2cm}
\begin{center}
\textbf{MT2: Automated 3D reconstruction from satellite images}\\
Organizer: Gabriele Facciolo, Centre de math\'ematiques et de leurs applications [CMLA] - Ecole Normale Sup\'erieure de Cachan, France\\
with\\ Carlo de Franchis and Enric Meinhardt-Lopis - Ecole Normale Sup\'erieure de Cachan, France
\end{center}
\vspace{1cm}
\begin{center}\textbf{Friday, June 8, \\9:30 AM - 11:30 AM \\Room B}\end{center}
%\vspace{2cm}
\begin{center}
\textbf{MT3: Regularization of Inverse Problem}\\
Organizer: Otmar Scherzer, Computational Science Center, University of Vienna, Austria
\end{center}

\newpage
%---------MT1
\begin{center}\textbf{Wednesday, June 6, 9:30 AM - 11:30 AM, Room B}\end{center}
%\vspace{2cm}
\begin{center}
\textbf{MT1: Computational Uncertainty Quantification for Inverse Problems}\\
Organizer: John Bardsley, Department of Mathematical Sciences, The University of Montana, MT, USA
\end{center}
%\vspace{2cm}

\begin{wrapfloat}{figure}{o}{0pt}
\includegraphics[scale=0.98]{John_Bardsley.jpg}
\end{wrapfloat}
Dr. Johnathan M. Bardsley is Professor of Mathematics at the University of Montana (UM) in Missoula, Montana, USA. He received his PhD from Montana State University in 2002 under the direction of Professor Curtis R. Vogel, with a dissertation focused on computational inverse problems. He then spent one year as a post-doc at the Statistical and Applied Mathematical Sciences Institute, under the direction of Professor H. Thomas Banks. He began his current job at UM in 2003, and since then has spent two years abroad as a visiting Professor: first at the University of Helsinki in Finland in 2006-07; and then at the University of Otago in New Zealand in 2010-11. Dr. Bardsley has published over 45 refereed journal articles and has given many presentations on his research around the world. He also organized Montana Uncertainty Quantification, a conference/workshop that took place at UM in June 2015. Dr. Bardsley's current research is focused, broadly, on uncertainty quantification for inverse problems, and more specifically, on the development of Markov chain Monte Carlo methods for sampling from posterior distributions that arise in both linear and nonlinear inverse problems. He has a forthcoming book, titled Computational Methods and Uncertainty Quantification for Inverse Problems, that will be published by SIAM.\\

\textbf{Abstract:} 
The field of inverse problems is fertile ground for the development of computational uncertainty quantification methods. This is due to the fact that, on the one hand, inverse problems involve noisy measurements, leading naturally to statistical (and hence uncertainty) estimation problems. On the other hand, inverse problems involve physical models that, upon discretization, are known only up to a high-dimensional vector of parameters, making them computationally challenging. Estimating a high-dimensional parameter vector in a discretized physical model from measurements of model output defines computational inverse problems. Such problems are typically unstable in that the estimates don?t depend continuously on the measurements. Regularization is a technique that provides stability for inverse problems, and in the Bayesian setting, it is synonymous with the choice of the prior probability density function. Once a prior is chosen, the posterior probability density function results, and it is the solution of the inverse problem in the Bayesian setting. The posterior maximizer ? known as the MAP estimator ? provides a stable estimate of the unknown parameters. However, uncertainty quantification requires that we extract more information from the posterior, which often requires sampling. The posterior density functions that arise in typical inverse problems are high-dimensional, and are often non-Gaussian, making the corresponding sampling problems challenging. In this mini-tutorial, I will begin with a discussion of inverse problems, move on to Bayesian statistics and prior modeling using Markov random fields, and then end with a discussion of some Markov chain Monte Carlo methods for sampling from posterior density functions that arise in inverse problems. 


%----MT2
\vspace{2cm}
\begin{center}\textbf{Thursday, June 7, 9:30 AM - 11:30 AM, Room B}\end{center}
%\vspace{2cm}
\begin{center}
\textbf{MT2: Automated 3D reconstruction from satellite images}\\
Organizer: Gabriele Facciolo, Centre de math\'ematiques et de leurs applications [CMLA] - Ecole Normale Sup\'erieure de Cachan, France\\
with\\ Carlo de Franchis and Enric Meinhardt-Lopis - Ecole Normale Sup\'erieure de Cachan, France
\end{center}

%\vspace{2cm}
\begin{wrapfloat}{figure}{o}{0pt}
\includegraphics[scale=0.4]{gabriele_facciolo.jpg}
\end{wrapfloat}
Gabriele Facciolo received his B.Sc. and M.Sc. in computer science from Universidad de la Republica del Uruguay, and his Ph.D. (2011) from Universitat Pompeu Fabra under the supervision of Vicent Caselles. During his thesis he contributed to a pioneering mathematical formalization of the image inpainting problem, and a formulation of temporally consistent video editing robust to illumination changes. He joined Jean-Michel Morel's group at the \'{E}cole Normale Sup\'erieure Paris-Saclay in 2011 where he is currently associate research professor. He has participated in many industrial projects, creating image processing algorithms and transferring technology with the CNES, Schlumberger, DxO Labs, and the foundation BarcelonaMedia. He has more than ten years of experience designing algorithms for remote sensing applications and collaborating with the French Space Agency (CNES) as part of the MISS project (Mat\'ematiques de l'Imagerie St\'er\'eoscopique Spatiale). The 3D reconstruction algorithms and the satellite stereo pipeline (github.com/MISS3D/s2p) he and his team have created within the CMLA have been adopted as the CNES?s official stereo pipeline. He and his team also won the 2016 IARPA Multi-View Stereo 3D Mapping Challenge. He is one of the founding Editors of IPOL (www.ipol.im), the first journal publishing articles associated to online executable algorithms.\\

\textbf{Abstract:} 
Commercial spaceborne imaging is experiencing an unprecedented growth both in size of the constellations and resolution of the images. This is driven by applications ranging from geographic mapping to measuring glacier evolution, or rescue assistance for natural disasters. For all these applications it is critical to automatically extract and update elevation data from arbitrary collections of multi-date satellite images. This multi-date satellite stereo problem is a challenging application of 3D computer vision: images are taken at very different dates, from very different points of view, and under different lighting conditions. The case of urban scenes adds further difficulties because of occlusions and reflections.
This tutorial is a hands-on introduction to the manipulation of optical satellite images, using complete examples with python code. The objective is to provide all the tools needed to process and exploit the images for 3D reconstruction. We will present the essential modeling elements needed for building a stereo pipeline for satellite images. This includes the specifics of satellite imaging such as pushbroom sensor modeling, coordinate systems, and localization functions. Then we will review the main concepts and algorithms for stereovision and tailor them to the case of satellite images. Finally, we will bring together these elements to build a 3D reconstruction pipeline for multi-date satellite images.
%-----MT3
\newpage
\begin{center}\textbf{Friday, June 8, 9:30 AM - 11:30 AM, Room B}\end{center}
%\vspace{2cm}
\begin{center}
\textbf{MT3: Regularization of Inverse Problem}\\
Organizer: Otmar Scherzer, Computational Science Center, University of Vienna, Austria
\end{center}

%\vspace{2cm}
\begin{wrapfloat}{figure}{o}{0pt}
\includegraphics[scale=0.35]{otmar_scherzer.jpg}
\end{wrapfloat}
Otmar Scherzer received his PhD and Habilitation from the University of Linz (Austria) in 1990, 1995, respectively. He was a postdoc researcher at Texas A\&M University and the University of Delaware. He held professorships at the Ludwig Maximilian University Munich, University of Bayreuth, University of Innsbruck before he became professor at the University of Vienna, where he is now the head of the Computational Science Center. In addition he is research group leader of the ``Imaging and Inverse Problems Group'' of the Radon Institute of Computational and Applied Mathematics (RICAM) in Linz, which is an institute of the Austrian Academy of Sciences. Otmar Scherzer is an expert in regularization theory and mathematical imaging. He has about 200 publications in leading journals in these fields and is editor of about 10 journals and book series, including SIAM J. imaging Sciences. Moreover, he published two monographs, and edited several books, including the Handbook of Mathematical Imaging in three volumes. In 1991 he received the Theodor K\"orner Prize, the Prize of the Austrian Mathematical Society, the science prize of Tyrol, and in 1999 the START-prize of the Austrian Science Foundation, which is the highest award for young Austrian scientists in Austria. From 2010 to 2017 he has been Vice-president of the Inverse Problems International Association (IPIA).\\

\textbf{Abstract:} 
Inverse Problems is an interdisciplinary research area with profound applications in many areas of science, engineering, technology, and medicine.
Nowadays, a core technology for solving imaging problems are
regularization methods. The foundations of these approximation methods were laid by Tikhonov decades ago, when he generalized the classical definition of well-posedness. 
In the early days of regularization methods, they were analyzed
mostly theoretically, while later on numerics, efficient solutions, 
and applications of regularization methods became important.
This MT gives a survey on theoretical developments in regularization 
theory: Starting from quadratic regularization methods for linear ill-posed 
problems, to convex regularization, and to non-convex regularization methods 
of non-linear problems. 
The theoretical analysis will be supported by particular 
imaging examples.
